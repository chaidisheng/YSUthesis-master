% !Mode:: "TeX:UTF-8"
\begin{conclusion}
\label{chap:conclusion}
在解决编码衍射模型中的相位恢复问题时,传统的优化算法存在运行时间久、效果不佳等缺点。其次对于大规模编码衍射图案,已有的相位恢复算法无法在较短时间内进行恢复。再者,压缩相位恢复算法在过低欠采样率下重构图像质量不佳。受非凸优化理论与深度学习技术的启发,本文基于共识均衡与一阶随机优化对编码衍射模型中的相位恢复问题进行建模,并利用有效的求解算法对优化方程进行求解,最后利用深度图像先验融合RED正则项解决了过低采样率下的压缩相位恢复问题。本文主要结论如下:

(1)基于共识方程的编码衍射成像算法基本解决了编码衍射模型中的相位恢复问题。TACE算法将编码衍射模型对应数据保真项的近邻算子与盲去噪算子置于共识方程之中进行相位恢复。该算法迭代致使数据保真项的近邻算子与盲去噪器趋于纳什均衡点。在理论层面,本文将MACE算法的收敛性假设条件推广为假设\ref{assumption:3-2}并给出了严格的数学证明。数值实验结果表明在重构实图像时,该算法能较好的保留图像的纹理、细节等信息,具有明显的优势,但性能略低于PnP-FBS算法。

(2)基于一阶随机优化的的加速编码衍射成像算法基本解决了面向大规模编码衍射图案的相位恢复问题。该算法根据观测值的可分离特征,利用一阶随机不动点算法进行求解,每次迭代随机选取编码衍射图案中的一个子集计算数据保真项的梯度。数值实验结果表明该算法在大规模编码衍射图案下具有良好的加速性能,但受限于已有的计算资源,SOCE算法并未发挥全部性能。

(3)基于深度图像先验融合RED正则项的压缩相位恢复算法基本解决了过低采样率下现有压缩相位恢复算法重建质量低的问题。该算法将显式的RED先验作为正则项添加到隐式的深度图像先验损失函数中,利用ADMM算法进行有效求解。实现过程中应用多线程技术并行化去噪迭代与网络参数更新以加快训练速度。数值实验结果表明在采样率为0.3、0.6、1.0下DPR-RED算法在MNIST数据集上的重构图像平均PSNR比已有的DPR算法分别高0.98db、0.48db、0.93db,在CelebA数据集上分别高1.32db、0.45db、0.49db,并且该算法对高斯与泊松噪声污染的观测值鲁棒。

本文虽然取得了上述成果,但在该领域仍有一些问题需要进一步研究与扩展:

(1)在TACE算法的基础之上引入其他有效的图像先验,进行算法性能的提升或使用强化学习进行深度展开(Deep Unrolling)训练选择该算法每次迭代的最优参数;

(2)SOCE算法对于高并发的GPU处理器,编码衍射图案数量级不足导致该算法无法发挥全部性能,后期考虑该算法的分布式实现;
 
(3)DPR-RED算法严重依赖于未训练神经网络的结构捕获图像低频先验信息,如何设计更加有效的网络结构有待进一步研究。
\end{conclusion} 